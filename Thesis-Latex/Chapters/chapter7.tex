% Chapter 7


\chapter{Summary and Future Work} % Write in your own chapter title
\label{Chapter7}
\lhead{Chapter 7. \emph{Summary and Future Work}} % Write in your own chapter title to set the page header
In this work, we suggested a direct algorithm for passive geolocation of a moving radio transmitter using an array of stationary receivers. This method is an extension of the conventional DPD method, to the case in which the transmitter is moving. While the conventional DPD method handles the stationary transmitter scenario, the suggested algorithm handles the scenario in which the transmitter is moving. \\
For the derivation of the algorithm, we defined a model for the scenario, and expressed the ML cost function for both the known and unknown signal scenario. We then explored the characteristics of the ML cost function , and developed its derivatives. Later, we suggested several algorithms for finding the maximum of the cost function, each relevant to a different scenario.\\
Later, we derived the Cramar-Rao lower bound of the scenario in order to compare the performance of the suggested algorithm to the theoretical bound.\\
Finally, we compared the performance of the suggested algorithm with the performance of conventional methods and with the lower bound using Monte-Carlo simulations. The suggested algorithm showed better performance than conventional methods, especially in lower SNR.\\

There are, of course, many unexplored matters, that were left out of the scope of this work. Future work could be held in the following directions:
\begin{itemize}
\item This work considered of estimating the position and velocity of the transmitter in only a single lag. In realistic scenarios, the moving transmitter is tracked during several lags. An interesting direction would be to develop an algorithm that samples the transmitter in several lags, and takes into consideration the motion model of the transmitter in order to achieve better performance.
\item This work handled only a single moving transmitter. It would be interesting to explore the scenario in which there are several moving transmitters.
\item In this work, we assumed that certain parameters, such as the position of the receivers is known in an infinite accuracy. It would be interesting to explore the sensitivity of the estimation to errors in the position of the receivers.
\item We developed only the known signals Cramer-Rao lower bound. Treating the signals as unknown stochastic signals and developing the unknown signals lower bound is an interesting direction.
\item In this work, we developed a method to analyse the performance of the algorithm based on the scenario parameters and the transmitted signal. It would be interesting to develop a method to find the optimal transmitted signal that would give the best estimation performance for a given scenario. This could be very helpful for developing beacon signals.
\end{itemize}
